\documentclass[12pt,a4paper]{article}
\topmargin -1.6cm
\addtolength{\textheight}{4cm}
\textwidth  15.5cm

\leftmargin      5mm
\rightmargin     5mm
\oddsidemargin   5mm
\evensidemargin  5mm

\usepackage{hyperref}
\usepackage{polski}
\usepackage[utf8]{inputenc}
\usepackage{graphicx}
\usepackage{units}
\usepackage{sty/style}

\projekt{aaaaaaaa}
\autor{Marcin Bober, 249426}
\przedmiot{Projekt specjalnościowy ARR}
\prowadzacy{Dr inż. Mirela Kaczmarek}

\begin{document}
\pdfpageheight   297mm
\pdfpagewidth    210mm

\StronaTytulowa
\SpisTresci

\pagebreak

\section{Oblicz macierze transformacji pomiędzy kolejnymi układami współrzędnych zgodnie z tabelą 1}

%   \begin{figure}[ht]
%     \centering
%     \includegraphics[width=0.86\textwidth]{img/img1.png}
%     \caption{Macierze transformacji}
%   \end{figure}

%   \begin{figure}[H]
%     \centering
%     \includegraphics[width=1\textwidth]{img/img3.png}
%     \caption{Macierze transformacji}
%   \end{figure}

%   \section{Wyznacz kinematykę manipulatora w SE(3)}
  
%   \begin{figure}[H]
%     \centering
%     \includegraphics[width=1\textwidth]{img/img2.png}
%     \caption{Kinematyka manipulatora}
%   \end{figure}

%   \begin{figure}[H]
%     \centering
%     \includegraphics[width=1\textwidth]{img/img4.png}
%     \caption{Kinematyka manipulatora}
%   \end{figure}

  \section{Zastanów się, które elementy są znaczące przy wyznaczaniu parametrów geometrycznych manipulatora i zdefiniuj parametryczną postać
  kinematyki potrzebną do wyznaczenia parametrów geometrycznych robota}
  
  Wyznaczanie parametrów geometrycznych manipulatora opiera się o 
  wykorzystanie wektora translacji znajdującego się w ostatniej kolumnie
  macierzy kinematyki. Informuje on o przsunięciu efektora względem 
  początku układu współrzędnych. Oprócz wektora translacji niezbędny będzie
  przykładowy zbiór współrzędnych przegubowych i współrzędnych efektora,
  aby ich podstawie móc wyliczyć parametry geometryczne manipulatora.
  
  \section{Na podstawie wyznaczonej kinematyki i danych pomiarowych wyznacz
  parametry geometryczne robota}
  
%   \begin{figure}[H]
%     \centering
%     \includegraphics[width=0.9\textwidth]{img/img5.png}
%     \caption{Parametry geometryczne}
%   \end{figure}

  \section{Przy jakich konfiguracjach jesteśmy w stanie odczytać bezpośrednio
  długości ramion?}

%   \begin{itemize}
%     \item q1 = -90
%     \item q2 = 0
%     \item q3 = 0
%     \item q4 = 90
%     \item q5 = 90
%     \item q6 = 0
%   \end{itemize}

%   X = d5 = 80mm

%   \begin{figure}[H]
%     \centering
%     \includegraphics[width=0.9\textwidth]{img/sym1.png}
%     \caption{Odczytywanie d5}
%   \end{figure}


%   \begin{itemize}
%     \item q1 = -90
%     \item q2 = 90
%     \item q3 = 90
%     \item q4 = 90
%     \item q5 = 90
%     \item q6 = 0
%   \end{itemize}

%   Z = d4 = 320mm

%   \begin{figure}[H]
%     \centering
%     \includegraphics[width=1\textwidth]{img/sym2.png}
%     \caption{Odczytywanie d4}
%   \end{figure}


%   \begin{itemize}
%     \item q1 = -90
%     \item q2 = 90
%     \item q3 = 90
%     \item q4 = 90
%     \item q5 = 90
%     \item q6 = 0
%   \end{itemize}

%   Z = d2 + d4 + d5 = 300mm + 320mm + 80mm

%   \begin{figure}[H]
%     \centering
%     \includegraphics[width=1\textwidth]{img/sym3.png}
%     \caption{Odczytywanie d2}
%   \end{figure}

\end{document}